\setlength{\baselineskip}{0.5cm}
\chapter{Benutzeroberfl�che}
\label{oberflaeche}
\begin{figure}
	\centering
	\includegraphics[width=1\textwidth]{startfenster.png}
	\caption{Admin-Ansicht der Startseite (Sichtbare Bereiche je nach Berechtigung) }
	\label{oberflaeche}
\end{figure}

\section{Administrationsbereich \normalsize{(bei entsprechender Berechtigung)}}
\begin{description}
	\item [Konto:] (siehe Kapitel \ref{admin_konto}) �bersicht �ber alle vorhandenen Konten; bearbeiten und l�schen von Konten.
	\begin{description}
		\item [Neu:] Neues Konto anlegen.
		\item [Zusammenlegen:] Zum Zusammenfassen doppelt angelegter Konten.
	\end{description}
	\item [Kostenstelle:] (siehe Kapitel \ref{admin_kostenstelle}) �bersicht �ber alle vorhandenen Kostenstellen; bearbeiten und l�schen von Kostenstellen; Zuordnen von Konten zu Kostenstellen.
	\begin{description}
		\item [Neu:] Neue Kostenstelle anlegen.
		\item [Zusammenlegen:] Zum Zusammenfassen doppelt angelegter Kostenstellen.
  	\item [Budgeteingabe:] Zur Eingabe des Budgets pro Kostenstelle und Gesch�ftsjahr.
  \end{description}
\end{description}

\section{Benutzerbereich}
\begin{description}
 \item [Bestellung:](siehe Kapitel \ref{bestellung})
 \begin{description}
 	\item [Neu:] Zum Anlegen einer neuen Bestellung.
 	\item [Suchen:] Zum Suchen und anschlie�enden Bearbeiten einer vorhandenen Bestellung.
 \end{description}
 \item [Rechnung:](siehe Kapitel \ref{rechnung})
 \begin{description}
 	\item [Neu:] Zum Anlegen einer neuen Rechnung.
 	\item [Suchen:] Zum Suchen und anschlie�enden Bearbeiten einer vorhandenen Rechnung.
 \end{description}
 \item [Personensuche:] F�r die Suche nach Personen (Wie auf der CIS-Seite)
 \item [Firma:](siehe Kapitel \ref{firma})
 \begin{description}
 	\item [Neu:] Zur Neuanlage einer Firma.
 	\item [Suchen:] Zum Suchen einer bereits eingetragenen Firma.
 \end{description}
\end{description}

\section{Berichte}
\begin{description}
	\item [Kostenstelle:](siehe Kapitel \ref{bericht_kostenstelle}) �berblick �ber alle Kostenstellen mit Bestell- und Rechnungssumme und dem Restbudget.
	\item [Tags:](siehe Kapitel \ref{bericht_tags}) Auswertung und Berechnung abh�ngig von den zugeordneten Tags.
\end{description}