\setlength{\baselineskip}{0.5cm}
\chapter{Administration}

Der Bereich der Administration ist nur bei entsprechender Berechtigung sichtbar.
Hier k�nnen Sie Konten und Kostenstellen anlegen und bearbeiten.

\section{Konto}
\label{admin_konto}

Durch einen Klick auf \textit{Konto} gelangen Sie zu einer �bersicht �ber alle derzeit angelegten Konten.
Mithilfe der Sybmole in der ersten Spalte k�nnen Sie einzelne Konten bearbeiten oder l�schen.

\subsection{Neu}
Der Men�punkt \textit{Neu} �ffnet ein Formular, in dem Sie ein neues Konto anlegen k�nnen.

\subsection{Zusammenlegen}
Die Praxis zeigt, dass gelegentlich �berfl�ssige Konten angelegt werden (zum Teil unbeabsichtigt), die in dieser Oberfl�che zusammengelegt werden k�nnen.
Die Oberfl�che zeigt zwei idente Listen, aus denen jeweils das �berfl�ssige (wird gel�scht) und das korrekte (bleibt erhalten) Konto durch klicken auf den Radio-Button am Ende jeder Zeile markiert wird. Anschlie�end werden die beiden Konten durch bet�tigen des Buttons \textit{->} vereint.

\section{Kostenstelle}
\label{admin_kostenstelle}
Durch einen Klick auf \textit{Kostenstelle} gelangen Sie zu einer �bersicht �ber alle derzeit angelegten Kostenstellen.
Mithilfe der Sybmole in der ersten Spalte k�nnen Sie einzelne Konten bearbeiten oder l�schen.
Au�erdem k�nnen Sie hier jeder Kostenstelle die entsprechenden Konten zuweisen.

\subsection{Neu}
Der Men�punkt \textit{Neu} �ffnet ein Formular, in dem Sie eine neue Kostenstelle anlegen k�nnen.

\subsection{Zusammenlegen}
Die Praxis zeigt, dass gelegentlich �berfl�ssige Kostenstellen angelegt werden, die in dieser Oberfl�che zusammengelegt werden k�nnen.
Sie sehen hier zwei idente Listen, aus denen jeweils die �berfl�ssige (wird gel�scht) und die korrekte (bleibt erhalten) Kostenstelle durch klicken auf den jeweiligen Radio-Button markiert wird. Anschlie�end werden die beiden markierten Kostenstellen durch bet�tigen des Buttons \textit{->} vereint.

\subsection{Budgeteingabe}
Hier k�nnen Sie f�r jede Kostenstelle das verf�gbare Budget im jeweiligen Gesch�ftsjahr eintragen.
W�hlen Sie aus dem Drop-Down Men� das gew�nschte Gesch�ftsjahr und klicken Sie auf \textit{Anzeigen}
Danach k�nnen Sie bei den gew�nschten Kostenstellen das Budget eingeben. Speichern Sie Ihre Eingaben mit dem Button am Ende der Liste.

