\setlength{\baselineskip}{0.5cm}
\chapter{Begriffserkl�rung}
\label{begriffe}

	\minisec{Konto}
Ein Konto dient der Darstellung von Gesch�fts- bzw. Verwaltungsvorf�llen.
Es verwaltet nur die reinen Ein- und Ausg�nge, also werden dort weder Gewinne noch Verluste ausgewiesen.

	\minisec{Kostenstelle}
Bezeichnet den Ort der Kostenentstehung und der Leistungserbringung. Sie wird nach Verantwortungsbereichen, r�umlichen, funktionalen, aufbauorganisatorischen oder verrechnungstechnischen Aspekten gebildet. Beispiele f�r funktionale Kostenstellen sind Materialkostenstellen, Fertigungskostenstellen, Forschungs- und Entwicklungskostenstellen, Verwaltungskostenstellen oder Vertriebskostenstellen.

	\minisec{Tag}
(['t{\ae}g], zu engl. �Etikett�)
Tag bezeichnet ein Schlagwort zur thematischen Einordnung eines Objekts. Ein Tag kann frei gew�hlt werden. Es dient dem leichteren Auffinden von Bestellungen bzw. der sinnvollen Zuordnung zu Themenbereichen. Es k�nnen sowohl einer ganzen Bestellung als auch einzelnen Posten innerhalb einer Bestellung Tags zugeordnet werden.\\
	Bei der Eingabe eines Tags, werden automatisch �bereinstimmungen mit vorhandenen Tags vorgeschlagen. W�hlen Sie entweder ein vorgeschlagenes Tag aus oder tippen Sie ein neues ein.
	Sie k�nnen auch mehrere Tags zuweisen. Trennen Sie diese mit einem Strichpunkt (;). Vermeiden Sie nach M�glichkeit Leer- und Sonderzeichen bei der Eingabe von Tags.
	
	\info{Wenn Sie bei einer Bestellung oder Bestellposition mehrere Tags definiert haben, werden diese auch bei der statistischen Auswertung mehrmals ber�cksichtigt und zur Gesamtsumme addiert. Zus�tzliche Tags verf�lschen also die Gesamtsumme.}
	
Verwendungsbeispiele:

\newcounter{fig}
\begin{list}{\textbf{Bsp. \arabic{fig}:}}{\usecounter{fig} \slshape}
\item Es wird B�romaterial bestellt. Als Tag definiere ich f�r die Gesamtbestellung den Begriff "`B�romaterial"'.
Bei den Berichten (siehe Kapitel \ref{bericht_tags}) wird nun die Summe aller Bestellungen, die als Tag "`B�romaterial"' definiert haben, aufgelistet.

\item In einer Bestellung bei der Firma XY wird ein Monitor und ein Textverarbeitungsprogramm bestellt.
Der Posten mit dem Monitor erh�lt den Tag "`Hardware"', der Posten mit dem Programm den Tag "`Software"'.
Bei den Berichten werden nun getrennt die Ausgaben f�r \textit{Hardware} und \textit{Software} erfasst.
\end{list}

\minisec{Bestellung}

Das Anlegen von Bestellungen dient in erster Linie der Vorausplanung und Vorab-Einteilung des Budgets.
Sie k�nnen das Bestellsystem auch nutzen, um sich einen �berblick �ber zuk�nftige Ausgaben zu verschaffen.
Solange die Bestellung nicht abgeschickt und freigegeben wurde, erfolgt auch keine tats�chliche Belastung des Budgets.

F�r einen besseren �berblick �ber die Ausgaben nutzen Sie besser die Rechnungssummen, da diese meist von der Bestellsumme abweichen (alte Katalogpreise, Skonti, Versandkosten,...).

Anschaffungen, die einen St�ckpreis von \EUR{250} �berschreiten, werden vom Inventarsystem erfasst.
Dabei werden die, bei den \textbf{Bestellungen} eingebenen Betr�ge ber�cksichtigt.
Wenn sich der tats�chliche Rechnungsbetrag vom Bestellbetrag unterscheidet, bessern Sie diesen bitte in der Bestellung aus,
um einen exakteren �berblick der vorhandenen Inventarwerte zu haben.

\minisec{Rechnung}

In den Rechnungen werden die tats�chlich gezahlten Betr�ge erfasst.
Bei den Berichten gibt Ihnen die Rechnungssumme also den genauen �berblick, wieviel vom Budget bisher tats�chlich belastet wurde.


