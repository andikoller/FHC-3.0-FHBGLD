\chapter{Installationsanleitung}
\label{Installationsanleitung - Client}
\section{Download von Seamonkey}
Zum Betrieb des FAS und Tempus wird ein Gecko basierter Browser ben�tigt. Die Applikation wurde auf der Seamonkey Suite entwickelt, l�uft aber auch unter MozillaSuite oder Firefox. Aktuelle Quellen des Seamonkey Browsers sind unter www.mozilla.org zu finden. Die Dialoge des Installationsassistenten k�nnen einfach mit 'Weiter' best�tigt werden. Der Standardpfad sollte unter Windows c:\textbackslash Programme\textbackslash mozilla.org\textbackslash seamonkey\textbackslash  sein.\\

\section{Download der Theme}
Die Theme(Skin) f�r Seamonkey ist im Grunde egal, jedoch unterst�tz die Calssic-Theme keine f�rbigen Buttons und teilweise kein Drag and Drop. Am besten eignet sich 'Orbit 3+1', es reicht aber auch die mitgelieferte Theme 'Modern'. Die Orbit 3+1 Theme ist im SVN unter portal/trunk/vilesci/admin/XPI/orbit-1.8f5-MiK.xpi zu finden. Zur Installation reicht es, das File in das Seamonkey Fenster zu ziehen und den aufscheinenden Dialog zu best�tigen.

\section{Einstellungen unter about:config}
Damit die Applikation reibungslos funktioniert m�ssen einige Sicherheitseinstellungen eingetragen werden.
�ffnen Sie dazu ein neues Browserfenster und geben Sie in die Adresszeile 'about:config' ein.\\
Im angezeigten Fenster m�ssen die folgenden Einstellungen ge�ndert werden.\\
- browser.cache.check\_doc\_frequency 1\\
- browser.cache.disk.capacity 0\\
- browser.downloadmanager.behavior 1\\
- dom.allow\_scripts\_to\_close\_windows true\\
- dom.disable\_window\_open\_feature.status false\\\
- signed.applets.codebase\_principal\_support true\\

Falls zum Versenden von Mails ein externer Mailclient verwendet wird (zB Outlook), muss ein neuer Wert unter about:config eingetragen werden.\\
Unter 'about:config' mit der rechten Maustaste ins leere klicken und new->boolean w�hlen:\\
network.protocol-handler.external.mailto true\\

\section{Seamonkey Bilder �ndern}
Um beim Starten der Applikation statt dem Seamonkey Logo das FH-Complete Logo anzuzeigen muss im Ordner c:\textbackslash Programme\textbackslash mozilla.org\textbackslash Seamonkey\textbackslash die Datei seamonkey.bmp mit dem FH-Complete Logo �berschrieben werden.\\
Um die Icons in der Titelleiste der Application zu �ndern, m�ssen die beiden Dateien tempus.ico und fas.ico (zu finden im SVN unter portal/trunk/skin/images/ )in den Ordner  c:\textbackslash Programme\textbackslash mozilla.org\textbackslash seamonkey\textbackslash chrome\textbackslash icons\textbackslash default\textbackslash  kopiert werden.

\section{Chrome Registrierung}
Ab Seamonkey und Firefox 1.5 m�ssen Applikationen die �ber .htaccess authentifizieren intern registriert werden. Hierzu gen�gt in klick auf https://vilesci.technikum-wien.at/vilesci/admin/XPI/FASoProduktiv/FASonline.xpi und dann auf Install.\\

\section{Verkn�pfung erstellen}
Zum Testen in der Adressleiste folgendes eingeben: chrome://fasonline/content/fasonline.xul\\

Wenn die Applikation ordnungsgem�� funktioniert, kann eine Verkn�pfung am Desktop angelegt werden:\\
C:\textbackslash Programme\textbackslash mozilla.org\textbackslash seamonkey\textbackslash seamonkey.exe -chrome chrome://fasonline/content/fasonline.xul

\section{Wichtige Hinweise}
FAS Benutzer m�ssen in der LDAP Gruppe HADESADM sein.