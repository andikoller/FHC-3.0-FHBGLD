\chapter{Berechtigung}
\label{Berechtigung}
Seit FHComplete 2.0 gibt es ein neues Berechtigungskonzept. Berechtigungen werden nun auf Organisationseinheiten aufgeh�ngt.
Wenn Personen Rechte auf eine Organisationseinheit zugeteilt bekommen, gelten diese auch
automatisch f�r die untergeordneten Organisationseinheiten. 
\section{Einzelberechtigungen}
Die Zuteilung der Berechtigungen f�r Personen erfolgt auf der Vilesci Seite unter Stammdaten->Berechtigungen\\
Berechtigungen k�nnen den Personen direkt zugeordnet werden. Hierbei muessen folgende Attribute ausgefuellt werden:\\
- berechtigung\_kurzbz\\
- uid\\
- oe\_kurzbz\\
- art\\
In diesem Fall bleiben die Felder rolle\_kurzbz und funktion\_kurzbz leer
\section{Rollenberechtigung}
Berechtigungen k�nnen zu Rollen zusammengefasst werden, um eine leichtere Verwaltung zu erm�glichen. (Assistenz, Adminstrator)
bei der Zuteilung einer Berechtigung zu einer Rolle muss die Art (in tbl\_rolleberechtigung) eingetragen werden. Die Tats�chlich 
verwendete Art, ist die Schnittmenge aus tbl\_rolleberechtigung.art und tbl\_benutzerrolle.art.\\
\\
Bei der Zuteilung von Rollen zu Personen sind folgende Felder auszuf�llen:\\
- rolle\_kurzbz\\
- uid\\
- oe\_kurzbz\\
Die Felder berechtigung\_kurzbz und funktion\_kurzbz bleiben leer.
\section{Funktionsberechtigung}
Zus�tzlich k�nnen Rechte Aufgrund von Benutzerfunktionen vergeben werden, um Beispielsweise alle Fachbereichskoordinatoren
bestimmte Rechte zuzuordnen.
Hier muss statt der UID die funktion\_kurzbz eingetragen werden. Das Feld oe\_kurzbz bleibt leer.
Bei der Pr�fung der Rechte, wird die oe\_kurzbz aus der Tabelle tbl\_benutzerfunktion herangezogen. Die Funktionen Mitarbeiter und Student sind nicht
extra in der Tabelle benutzerfunktion eingetragen sondern werden automatisch zugewiesen.\\
Die Zuteilung der Funktionsberechtigungen erfolgt auf der Vilesci Seite unter Personen->Funktionen->Berechtigung\\
\section{Negativrechte}
Alle Berechtigungen und Rollen k�nnen auch als Negativrecht eingetragen werden. Hierbei muss das Feld negativ auf true gesetzt werden.
Dies kann dazu ben�tzt werden, um bei der Zuteilung einer gesamten Rolle, der Person wieder einzelne Rechte zu entziehen.

\section{Aufbau}
Berechtigungen sind nach einem bestimmten Schema aufgebaut:\\
\textbf{lehre/lehrveranstaltung}\\
\textbf{lehre/lehrveranstaltung:begrenzt}\\
Zuerst wird das Modul genannt in dem die Berechtigung zum Tragen kommt. Durch einen '/' getrennt folgt der Name der Berechtigung. Die Berechtigung kann durch ':' in Unterberechtigungen unterteilt werden. 

\achtung{\textbf{Wichtig} Wenn eine Person die Berechtigung f�r 'lehre/lehrveranstaltung' hat, dann hat sie auch automatisch die Rechte f�r die Berechtigung 'lehre/lehrveranstaltung:begrenzt'.}\\

\section{Webservice/SOAP Schnittstellen}
Da die Webserivce Schnittstellen auch f�r Studierendenprojekte verwendet werden, k�nnen hier die Rechte noch detaillierter verwaltet werden.
Die Leserechte der Webservices k�nnen auf einzlene Attribute eingeschr�nkt werden.
Die Zugriffssteuerung dazu erfolgt in der Datenbank �ber die Tabelle system.tbl\_webservicerecht.
Hier kann folgendes angegeben werden:\\
\\
- F�r welche Berechtigung wird das Webservicerecht erteilt (berechtigung\_kurzbz)\\
- Welche Webservice Methode betrifft das Recht (methode)\\
- Welches Attribut dieser Methode darf angezeigt werden (attribut)\\
\\
F�r jedes Attribut das angezeigt werden soll, ist eine eigene Zeile in die Tabelle einzuf�gen.
Attribute die nicht in dieser Tabelle aufscheinen, werden aus dem Webservice Response entfernt.\\
Auf diese Weise k�nnen eigene eingeschr�nkte Berechtigungen auf Webservices erstellt werden.\\
\\
