\chapter{BIS-Meldung}
\section{Mitarbeiter}
\subsection{Ablauf}
Es gibt die M�glichkeit, die BIS-Meldung des Vorjahres zu importieren. Dies ist Hilfreich, wenn die Meldung zum Ersten mal mit dem FAS erstellt wird, damit die Daten nicht h�ndisch eingetragen werden m�ssen. Dies geschieht �ber den Men�punkt BIS->Mitarbeiter->Import.\\
\\
�ber den Men�punkt BIS->Mitarbetier->checkVerwendung kann gepr�ft werden, ob alle Mitarbeiter eine g�ltige BIS-Verwendung eingetragen haben. Die aufscheinenden Fehler sollten vor der BIS-Meldung bereinigt werden.\\
\\
Wenn die Fehler in checkVerwendung behoben wurden, kann checkFunktion aufgerufen werden. Dieses Script erstellt die BIS-Funktionen. Hierbei wird die Anzahl der Stunden anhand der Lehrauftr�ge zu den Verwendungen hinzugef�gt.\\
\\
Nach diesen Schritten, kann die Meldung �ber den den Men�punkt 'Meldung generieren' erstellt werden.
\subsection{Inaktive Studieng�nge}
Bei der Personalmeldung d�rfen keine Studiengangsleiter von Studieng�ngen gemeldet werden, die keinen Unterricht mehr haben. Diese Studieng�nge k�nnen aber nicht immer deaktiviert werden (z.B. wenn noch Diplomanden vorhanden sind ohne Abschlusspr�fung). Diese Studieng�nge m�ssen in der Datei /vilesci/bis/personalmeldung.php in das Array \$nichtmelden eingetragen werden.