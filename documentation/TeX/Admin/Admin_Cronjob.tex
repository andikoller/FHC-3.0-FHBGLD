\chapter{Cronjob}
\label{Cronjob}
\\
Um die Daten regelm��ig zu Pr�fen und um Inkonsistenzen vorzubeugen, k�nnen im FH-Complete Zeitgesteuerte Scripte (Cronjobs) gestartet werden.
Cronjobs werden im FHComplete gesammelt verwaltet. Damit diese funktionieren, muss in der Crontab folgender Eintrag angelegt werden:\\
\\
*/5   *   * * *    /var/www/vilesci/htdocs/vilesci/cronjobs/cronjob.php > cron.log\\
\\
Dieser Eintrag muss auf jedem Server gemacht werden, auf dem ein Job laufen soll.\\
Die einzelnen Jobs k�nnen �ber die Vilesci-Seite unter Admin->Cronjobs verwaltet werden.
Zum Testen ob der Dienst funktioniert existiert ein Testjob. Dieser sendet automatisch ein Mail wenn der Job gestartet wird.
\\
\\
\textbf{Folgende Felder k�nnen zu Cronjobs eingetragen werden:}\\
- Titel: Titel des Jobs\\
- Beschreibung: Kurzbeschreibung zu dem Job\\
- Server: Server auf dem der Job ausgef�hrt werden soll (Crontab-Eintrag muss auf diesem Server vorhanden sein!)\\
- Datei: absoluter Pfad zu dem Cronjob. Beispiel f�r den Testjob: '/var/www/vilesci/htdocs/vilesci/cronjobs/testjob.php'\\
- Zeitangabe: der Ausf�hrungszeitpunkt kann entweder direkt oder als Interval angegeben werden. (mit */<Zeitinterval) Die Funktionsweise ist mit jener 
der Unix Cronjobs vergleichbar.\\
Beispiele:
Job soll jeden 1. des Monats um 02:00 Uhr gestartet werden:\\
Das Feld Tag muss auf 1 gesetzt werden, Stunde auf 2 und Minute auf 00. Die anderen Felder bleiben leer.\\
Job soll jeden 2. Tag laufen:\\
Das Feld Tag wird auf */2 gesetzt. Alle anderen bleiben leer.\\
Der Job soll jeden Sonntag im Jahr 2010 um 01:00 Uhr laufen:\\
Jahr wird auf 2010 gesetzt, Wochentag auf Sonntag, Stunde auf 1 und Minute auf 00. Die restlichen Felder bleiben leer.\\
- Aktiv: inaktive Jobs werden nicht ausgef�hrt\\
- Standalone: Wenn dies gesetzt ist, dann darf der Job nur alleine ausgef�hrt werden. Wenn zur selben Zeit ein anderer Job l�uft, dann wird dieser nicht ausgef�hrt.\\
- Reihenfolge: Bei Jobs die zur gleichen Zeit gestartet werden, werden die mit der niedrigeren Reihenfolge zuerst ausgef�hrt.\\
- Varialben: Hier k�nnen Variablen eingetragen werden, die an den Cronjob �bergeben werden. Diese m�ssen im JSON-Format eingetragen werden. Zu unterst�tzung steht ein Variablen-Editor zur Verf�gung.
Beim Anlegen eines neuen Jobs k�nnen die Variablen automatisch initialisiert werden. Dies ruft den Cronjob mit einem Initialisierungsparameter auf. Das Script setzt daraufhin die Standard-Variablen f�r diesen Job.
\\

