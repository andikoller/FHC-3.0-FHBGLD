\chapter{Erweiterbarkeit}
\label{Erweiterbarkeit}

Es gibt die M�glichkeit, das Programm zu erweitern, ohne dabei gr��ere Probleme bei Updates zu verursachen. \\
\\
Teile des Programmcodes wurden in das Verzeichnis /trunk/include/tw/ ausgelagert um individuelle Anpassungen am Programm durchzuf�hren. (zB Men�struktur des CIS, Vorlagen f�r PDFs die nicht �ber XSLT erstellt werden, ...)
Dieser Ordner kann kopiert und unter einem anderen Namen im include Ordner abgelegt werden. Der Pfad zu diesem Ordner wird im config.inc.php in
der Konstante EXT\_FKT\_PATH abgelegt. Danach k�nnen die Dateien in diesem Verzeichnis abge�ndert werden, ohne das diese bei einem Update �berschrieben werden.\\

\section{Zus�tzliche Men�punkte im FAS-Online}
Im FAS-Online k�nnen auch zustzliche Men�punkte eingef�gt werden, ohne dass diese bei einem Update �berschrieben werden. Dies ist n�tzlich falls zus�tzliche Statistiken o.�. eingebaut werden sollen. Dazu gibt es im Ordner /trunk/include/tw/ zwei Files:\\
- fas\_zusatzmenues.inc.php\\
- fas\_zusatzmenues.js.php\\
\\
Im ersten File kann der XUL Code f�r die Erstellung zus�tzlicher Men�s eingetragen werden.\\
Das Zweite File enth�lt den dazugeh�rigen Javascript Code um etwa die Seite mit der Statistik aufzurufen.\\
\\
Einen Beispielcode f�r das Hinzuf�gen eines neuen Men�punktes befindet sich in den oben genannten Dateien.\\
\\
Der neue Men�punkt wird im FAS, nach einf�gen des Codes, automatisch vor dem Men�punkt Extras eingebunden.