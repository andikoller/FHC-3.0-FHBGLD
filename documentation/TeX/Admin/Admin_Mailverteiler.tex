\chapter{Mailverteiler}
\label{Mailverteiler}
Die Mailverteiler werden t�glich neu erstellt. Hierzu werden Textfiles mit den entsprechenden Verteilern erstellt. Dies erfolgt durch die Scripten im Verzeichnis system/mlists.

Verteiler die Automatisch erstellt werden, haben in der Gruppe das Attribut generiert auf true gesetzt.\\
(Hier werden die Personen automatisch zugeteilt �ber das Script system/mlists/mlists\_generate.php)\\

\label(Sperrung von Mailverteilern)
Mailverteiler k�nnen gesperrt werden, damit diese nur von Mitarbeitern verwendet werden k�nnen.\\
Dazu muss im config die Konstante MAILVERTEILER\_SPERRE auf true gesetzt sein. Zur Benutzung des Verteilers muss dieser dann erst Freigeschalten werden. Dies erfolgt �ber das kleine Symbol links neben dem Verteilernamen. \\
Das Symbol zur Entsperrung eines Verteilers erscheint nur dann, wenn aktiv=false ist.\\

(Diese Schritte zeigen nur die Einstellungen im FH-Complete damit der Verteiler als gesperrt angezeigt wird. Die tats�chliche Sperre der Verteiler muss extra am MailServer eingerichtet werden.)\\

\section{Alias}
Vom System werden automatisch Email-Aliase angelegt. Diese werden mit vorname.nachname@domain generiert. Wenn der Alias bereits vergeben ist, wird kein Ersatz generiert.\\
Die Email-Aliase werden f�r alle Benuzter angelegt (auch f�r Studenten).\\
Das Erstellen von Aliase f�r Studenten kann auch unterbunden werden, wenn in der Datei /include/globals.inc.php der Studiengang f�r den keine Aliase erstellt werden sollen, zu dem Array 'noalias' hinzugef�gt wird.\\
\\
Die Aliase werden in die Datei tw\_alias.txt exportiert.