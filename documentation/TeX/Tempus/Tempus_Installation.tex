\chapter{Installation von Tempus}
\label{Installation}

Tempus ben�tigt einen XUL-Unterst�tzenden Browser im Hintergrund.

Es wird empfohlen, SeaMonkey oder Firefox von Mozilla zu installieren, da TEMPUS\raisebox{1ex}{\tiny \copyright} speziell daf�r konzipiert wurde.
Die Software erhalten Sie als Freeware auf www.mozilla.org.\\

\begin{itemize}
\item Installieren Sie den Browser und starten Sie diesen.
\item �ndern Sie im Men�punkt ''Ansicht'' das Theme auf \textit{Modern}
\item Geben Sie in der Adressenleise \textit{about:config} ein und setzen Sie den Eintrag \textit{signed.applets.codebase\_principal\_support} auf \textbf{true}.
\item Schlie�en Sie anschlie�end den Browser und legen Sie eine Verkn�pfung zu der .exe-datei des Browsers auf Ihrem Desktop oder in Ihrem Startmen� an.
\item Klicken Sie diese Verkn�pfung mit der rechten Maustaste an, w�hlen Sie \textit{Eigenschaften} und f�gen Sie dem Link-Ziel die Erweiterung \textit{-chrome} gefolgt von der ULR auf dem Webserver hinzu. Das Ziel k�nnte also Beispielsweise \textit{C:/Programme/mozilla.org/SeaMonkey/seamonkey.exe -chrome http://domain.at/tempus.xul.php} lauten.
\item Zuletzt bearbeiten Sie die Datei \textit{prefs.js}, die sich in den Anwendungsdaten Ihres Users unter Mozilla/Profiles/default/px5g6kxi.slt/prefs.js befindet. Der Pfad k�nnte also beispielsweise \textit{ C:/Dokumente und Einstellungen/username/Anwendungsdaten/Mozilla/Profiles/default/px5g6kxi.slt/prefs.js} lauten.\\
\end{itemize}

�ndern Sie dort die folgenden Eintr�ge, bzw. f�gen Sie sie hinzu:\\ 

user\_pref(''capability.principal.codebase.p0.granted'', ''UniversalXPConnect'');\\
user\_pref(''capability.principal.codebase.p0.id'', ''http://vilesci.technikum-wien.at'');\\
user\_pref(''capability.principal.codebase.p1.granted'', ''UniversalXPConnect'');\\
user\_pref(''capability.principal.codebase.p1.id'', ''https://vilesci.technikum-wien.at'');\\
user\_pref(''capability.principal.codebase.p2.granted'', ''UniversalXPConnect'');\\
user\_pref(''capability.principal.codebase.p2.id'', ''http://dav.technikum-wien.at'');\\


Zuletzt k�nnen Sie noch das Icon der Verkn�pfung und des Programmsymbols �ndern.
Das Icon der Verkn�pfung k�nnen Sie einfach mit der rechten Maustaste, Eigenschaften, Anderes Symbol �ndern.\\
Das Programmicon �ndert sich, wenn Sie die Datei Tempus.ico in den Ordner C:/Programme/mozilla.org/SeaMonkey/chrome/icons/default kopieren.
Eventuell unterscheidet sich Ihr Zielordner geringf�gig von diesem Beispiel.