\chapter{�ndern von bereits verplanten Stunden}

Ist es notwendig, eine �nderung von bereits verplanten Stunden vorzunehmen, m�ssen diese nicht unbedingt gel�scht und neu verplant werden.
Eine einfache �nderung wie beispielsweise einen Raumtausch oder eine Stundenverschiebung innerhalb der angezeigten Woche ist durch markieren und verschieben per Drag\&Drop leicht durchzuf�hren.

\section{�nderungen im selben Fenster}

Grunds�tzlich gen�gt es, eine Lehrveranstaltung direkt mit der Maus anzuklicken, die Maustaste gedr�ckt zu halten und die Lehrveranstaltung an ihren neuen Platz zu schieben. Dies setzt allerdings voraus, dass am neuen Termin der selbe Raum frei ist und auch der Lektor und die Studenten eine Terminl�cke haben. Sollte es zu einer Kollision kommen, erscheint (bei aktivierter Kollisions�berpr�fung) ohnehin eine Fehlermeldung.

\section{Verschieben nach Raumvorschlag}

Gew�hnlich ist es notwendig, bei einer Verschiebung den Raum zu wechseln.
Klicken Sie dazu mit der rechten Maustaste und dann auf Raumvorschlag auf die erste Lehreinheit der gew�nschten Lehrveranstaltung.
Im Fenster erscheinen alle freien R�ume, an allen m�glichen Terminen, unter Ber�cksichtigung aller beteiligten Lektoren und Studenten. Markieren Sie die Stunden die Sie verschieben m�chten und ziehen Sie diese auf den gew�nschten Raum.
Informationen zum Markieren finden Sie im Kapitel \ref{Markieren von Stunden}

\section{�nderungen �ber 2 Fenster}

Wird es notwendig, eine Lehrveranstaltung in eine andere Woche zu verschieben, muss dies �ber ein zweites Fenster geschehen. Verkleinern Sie dazu das ge�ffnete Fenster und �ffnen Sie TEMPUS\raisebox{1ex}{\tiny \copyright} erneut.
Ordnen Sie die Fenster unter- oder nebeneinander an, so dass Sie bequem �ber beide Fenster arbeiten k�nnen.
 
Bl�ttern Sie in einem Fenster zu der Lehrveranstaltung, die Sie verschieben m�chten. Bl�ttern Sie im 2. Fenster zu dem neuen Wunschtermin. Welche Fensterkombinationen Ihnen dabei zur Verf�gung stehen, entnehmen Sie bitte folgender Darstellung.

%\small
\begin{center}
\begin{tabular}{rcll}
AUS&&IN&FOLGE\\
\cline{1-4}
Verband&\begin{math}\rightarrow \end{math}& Raum&Raum�nderung der LV\\
Verband&\begin{math}\rightarrow \end{math}& (selber) Verband&Termin�nderung der LV\\
Verband&\begin{math}\rightarrow \end{math}& (selber) Lektor&Termin�nderung der LV\\
Lektor&\begin{math}\rightarrow \end{math}& Raum&Raum�nderung der LV\\
Lektor&\begin{math}\rightarrow \end{math}& Verband&Termin�nderung der LV\\
Raum&\begin{math}\rightarrow \end{math}& Raum&Raum�nderung der LV\\
Raum&\begin{math}\rightarrow \end{math}& Verband&Termin�nderung der LV\\
Raum&\begin{math}\rightarrow \end{math}& (selber) Lektor&Termin�nderung der LV\\
\end{tabular}
\end{center}
%\normalsize

\achtung{Achten Sie beim arbeiten mit 2 Fenstern darauf, dass in beiden Fenstern die richtige Kalenderwoche angezeigt wird. Wenn Sie in einem Fenster die Kalenderwoche �ndern, hat dies keine Auswirkungen auf die anderen ge�ffneten Fenster.}

\info{�ber 2 Fenster k�nnen die Lehrveranstaltungen nur einzeln verschoben werden. Das Markieren und Verschieben mehrerer LV ist nur im selben Fenster m�glich.}

Klicken Sie auf die Lehrveranstaltung, die Sie verschieben m�chten, halten Sie die Maustaste gedr�ckt.
Ziehen Sie die Lehrveranstaltung in das 2. Fenster an den gew�nschten Termin.
Lassen Sie die Maustaste los.
Klicken Sie im ersten Fenster auf den Button ''Aktualisieren'' \includegraphics{icon_aktualisieren}
