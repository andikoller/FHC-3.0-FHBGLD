\chapter{Tipps und Tricks}
\label{Kapitel Tipps}

\section{Planungsablauf}

Es empfiehlt sich, die Planung der Lehreinheiten mit den Vorlesungen zu beginnen, die alle Studenten gemeinsam besuchen. Das nachtr�gliche Einpflegen solcher Stunden ist schwieriger, als L�cken f�r Teilgruppen zu finden.

\section{Falsche Stundenblockung}

Angenommen, Sie wollen eine LV �ber 3 Lehreinheiten verplanen, es ist jedoch nur eine 2-Stunden Blockung eingestellt. Es ist dann nicht unbedingt notwendig, die Blockung zu ver�ndern. Verplanen Sie zun�chst 2 LE, l�schen Sie nur die zweite (rechte Maustaste ''Entfernen''), verplanen Sie erneut 2 LE direkt danach.

\section{R�ume umschlichten}

Angenommen, Sie wollen (z.B. bei Umschlichtungen in den R�umen) zwei Lehrveranstaltungen am selben Termin in den jeweils anderen Raum schieben. Da Ihnen kein Zwischenspeicher zur Verf�gung steht und eine solche Verschiebung eine Kollision verursacht, m�ssen Sie sich mit einem Trick behelfen. Verschieben Sie zun�chst die erste LV in das zweite Fenster aber an einen Termin, der aller Wahrscheinlichkeit nach frei ist (z.B. Samstag Abend). Verschieben Sie dann die zweite LV an den freigewordenen Termin. Korrigieren Sie zuletzt die erste LV an Ihren eigentlichen Platz.

\section{Kollisions�berpr�fung �berbr�cken}

Wenn Sie in der Registerkarte ''Lehrveranstaltung'' bei einer Lehreinheit die UNR manuell gleich mit einer anderen Lehreinheit setzen, kollidieren diese bei der Verplanung nicht.