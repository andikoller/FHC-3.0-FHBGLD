\chapter{Entwicklung-Software}
\section{Namensgebung}
Eine einheitliche Form der Namensgebung f\"ur Variablen, Konstanten und anderer Komponenten erleichter es dem Entwickler, den Code des anderen zu verstehen.

\subsection{Dateinamen}
Alle Dateinamen haben die f\"ur ihren Dateityp {\bf bestimmte Endung}. So haben z.B. HTML Dateien die Endung .html oder PHP Dateien .php. 
Es gibt jedoch Datentypen die eine {\bf eigene Schreibweise} haben. Hier eine Aufz\"ahlung der wichtigsten: \newline



\begin{tabular}{ll}
Klassen: & <klasse>.class.php \\
PHP-File: & <name>.php \\
HTML-File: & <name>.hmtl \\
RDF-File: & <name>.rdf \\
JavaScript: & <name>.js \\
CSS: & <name>.css \\
XUL: & <name>.xul \\
Config: & <name>.inc.php \\
\end{tabular}

Hierzu sind Endungen wie <name>.rdf.php oder <name>.js.php erlaubt. \newline
Es sind nur {\bf alphanumerische Zeichen, Underscores und Trennstriche} erlaubt. Beispiele f\"ur g\"ultige Namen sind: \newline
\begin{itemize}
\item studiengang.class.php
\item bestellung.rdf.php
\item tablesort.css
\item kontakt.js.php
\item ToDo\_CIS.html
\end{itemize}

\subsection{Variablen}
Alle Variablen m\"ussen mit {\bf Kleinbuchstaben beginnen} und der "`{\bf camelCaps}"' Namenskonventation folgen. Das bedeutet, wenn ein Variablenname aus mehereren
Namen besteht muss der Anfangsbuchstabe von jedem {\bf neuen Wort gro\ss{}geschrieben} werden. Repr\"{a}sentiert die Variable eine ID so wird die Endung ID mit \_ angef\"ugt. 
Variablennamen sind so {\bf kurz} und so {\bf verst\"andlich} wie m\"oglich
zu halten. Variablen wie "`\$i"' und "`\$j"' d\"urfen nur bei Loops verwendet werden.\\
{\bf Variablen in Klassen m\"ussen immer so hei\ss{}en wie sie in der Datenbank angelegt wurden.}
\begin{verbatim}
$anzahlVariablen
$datum
$person_id
\end{verbatim}

\subsection{Konstanten}
Pfade werden in Konstanten immer mit / beendet

\begin{verbatim}
define('SERVER_ROOT','http://calva.technikum-wien.at/');
\end{verbatim}

\subsection{Session Variablen}
Session Variablen die nur ein bestimmtes {\bf Modul} betreffen, werden folgenderma\ss en benannt: 
{\bf [Modulname] / [Name der Variable]}. Andere Session Variablen, die das ganze System betreffen werden ohne Modulnamen geschrieben. 
\begin{verbatim}
$_SESSION['cms/menu']
$_SESSION['wawi/user']
$_SESSION['user']
\end{verbatim}
Eine Session Variable wird {\bf immer klein geschrieben}. 

\subsection{Funktionen und Methoden}
Funktionsnamen m\"ussen immer mit einem {\bf Kleinbuchstaben beginnen}. Wenn eine Funktion aus mehreren Namen besteht, muss der Anfangsbuchstabe von jedem 
{\bf neuen Wort gro\ss geschrieben} werden. Funktionen und Methoden m\"ussen so {\bf klar} wie m\"oglich bezeichnet werden, damit anhand des Funktionsnamen jeder
versteht, wof\"ur diese bestimmt sind. Optionale \"Ubergabeparameter m\"ussen immer mit null initialisiert werden. \newline
Hier ein paar {\bf Beispiele} f\"ur g\"ultige Funktionsnamen:

\begin{verbatim}
getAllDetailsFromBestellung()
load()
saveTags($tag, $visible=null)
\end{verbatim}

\section{Kommentare}
Kommentare die nur eine Zeile lang sind sollen mit // beginnen. F\"ur Kommentare, die l\"anger als eine Zeile sind gilt als Start die /* Zeichenfolge und als Beendigung */. 

\subsection{Klassen}
Jede Klasse muss im Minimum einen {\bf Docblock} mit einer Beschreibung enthalten. \newline
Mit einem Docblock Kommentar bezeichnet man spezielle Kommentare, die sich automatisch generieren um PHP Abschnitte genauer zu kommentieren. Diese beginnen mit {\bf /**} und es k\"onnen spezielle tags wie @param oder @return benutzt werden. 

\begin{verbatim}
/**
 * Short description for class
 *
 * Long description for class (if any)...
 */
\end{verbatim}
 
\subsection{Funktionen und Methoden}
Jede Funktion oder Klassenmethode muss im Minimum einen {\bf Docblock} mit den folgenden Tags enthalten: {\bf Beschreibung, Parameter und R\"uckgabewerte}.
\begin{verbatim}
/**
 * Short description for the function
 *
 * Long description for the function (if any)...
 *
 * @param  array  $array  Description of array
 * @param  string $string Description of string
 * @return boolean 
 */
\end{verbatim}

\subsection{Variablen}
Jede Klassenvariable muss im Minimum einen {\bf Docblock} oder {\bf normalen} Kommentar enthalten welche den Typ der Variable ersichtlich macht. Wenn erforderlich kann dieser auch eine {\bf Beschreibung} der Variablen enthalten:
\begin{verbatim}
/**
 * Variable Description
 * @var array
 */
\end{verbatim}
\subsection{Programmkopf}

\begin{verbatim}
/* Copyright (C) 2011 Technikum-Wien
 *
 * This program is free software; you can redistribute it and/or modify
 * it under the terms of the GNU General Public License as
 * published by the Free Software Foundation; either version 2 of the
 * License, or (at your option) any later version.
 *
 * This program is distributed in the hope that it will be useful,
 * but WITHOUT ANY WARRANTY; without even the implied warranty of
 * MERCHANTABILITY or FITNESS FOR A PARTICULAR PURPOSE.  See the
 * GNU General Public License for more details.
 *
 * You should have received a copy of the GNU General Public License
 * along with this program; if not, write to the Free Software
 * Foundation, Inc., 59 Temple Place, Suite 330, Boston, MA 02111-1307,USA.
 *
 * Authors: Christian Paminger <christian.paminger@technikum-wien.at>,
 *     Andreas Oesterreicher <andreas.oesterreicher@technikum-wien.at> and
 *     Karl Burkhart <burkhart@technikum-wien.at>.
 */
\end{verbatim}
Zu beginn jedes Skriptes soll ganz oben der {\bf GNU Header} eingef\"ugt werden. \newline
Im {\bf Authors Block} werden nur Programmierer erw\"ahnt die das aktuelle File auch wirklich bearbeitet haben. 

\section{Programmierstil}
\subsection{Strings}
Ein String wird mit echo ausgegeben und sollte grunds\"atzlich mit {\bf einfachen Apostrophen}(single quotes) abgegrenzt werden. Dies hat den Vorteil, dass PHP Code im String {\bf nicht ausgef\"uhrt wird} und so z.B. HTML Ausgaben einfacher und schneller sind. Eine Folge von auszugebenden Strings kann mit Beistrich getrennt werden (anstatt mit . verkn\"upft), wodurch die Ausgabe beschleunigt wird

\begin{verbatim}
$message = 'Hello World';
echo '<div style="float: right;">',$message,'</div>';
echo '<div style="float: right;">'.$message.'</div>';
\end{verbatim}

\subsection{IF - ELSE}
Um Bl\"ocke zu gruppieren, k\"onnen {\bf zus\"atzliche Klammern} eingesetzt werden. Bei l\"angeren Bedingungen werden alle Bedingungen {\bf untereinander dargestellt}, um die Lesbarkeit zu erh\"ohen. Die \"offnende geschweifte Klammer wird immer in die {\bf n\"achste separate Zeile geschrieben}. Die abschlie\ss ende geschweifte Klammer erh\"alt ebenfalls immer eine separate Zeile. Jeder Code innerhalb der geschweiften Klammern muss {\bf einen Tabulator} (standardm\"a\ss ig vier Leerzeichen) einger\"uckt werden. \begin{verbatim}
if($user == 'Herbert')
{
   echo 'Hallo Herbert';
}
else
{
   echo 'Falscher User';
} 
\end{verbatim}
Bei sehr kurzen "`if-else"' Konstrukten kann auch der {\bf dreifach konditionale Operator} verwendet werden:
\begin{verbatim}
$sampleVar = isset($_GET['sampleVar']) ? $_GET['sampleVar'] : '';
\end{verbatim}
Gibt es nur eine ausf\"uhrende Zeile, so k\"onnen die geschweiften Klammern auch {\bf weggelassen werden}.
\begin{verbatim}
if($user == 'Herbert')
   echo 'Hallo Herbert';
else
   echo 'Falscher User';
\end{verbatim}

\subsection{Datenbankabfragen}
\begin{itemize}
\item Datenbankabfragen sind grunds\"atzlich \"uber die Datenbankklasse abzusetzen. 
\item Um SQL-Injections zu verhindern, sind die Paramter mittels der Funktion db\_add\_param zu escapen. 
(M\"ogliche \"Ubergabeparameter sind FHC\_INTEGER, FHC\_STRING und FHC\_BOOLEAN)
\item SQL Statements sind mit Strichpunkt abzuschlie\ss{}en.
\item Vor jedem Tabellennamen ist das entsprechende Schema anzugeben
\item Schl\"usselw\"orter (zB SELECT, WHERE, FROM, GROUP BY) sind gro\ss{} zu schreiben
\end{itemize}
\begin{verbatim}
$qry = "SELECT * FROM public.tbl_person WHERE 
person_id=".$db->db_add_param($person_id, FHC_INTEGER, false).';';
\end{verbatim}

Boolean Attribute sind ber die Datenbankklasse zu parsen um inkompatibilit\"aten mit verschiedenen DB-Systemen zu verhindern.

\begin{verbatim}
$aktiv = $db->db_parse_bool($row->aktiv);
\end{verbatim}

\subsection{HTML-Ausgaben}
HTML Attribute sollen generell mit doppelten Hochkomma umschlossen werden.
Variablen die innerhalb des HTML-Codes ausgegeben werden sind mit der Funktion convert\_html\_chars zu escapen. Dies ist vorallem wichtig bei Strings die den Typ Text oder varchar in der Datenbank haben. 

\begin{verbatim}
echo '<div style="float: right;">',$basis->convert_html_chars($message),'</div>';
\end{verbatim}


\subsection{Sonstiges}
\begin{itemize}
	\item {\bf Geschwungene Klammern} \\
	Um einen Block zu definieren werden die geschwungenen Klammern {\bf immer} in der neuen Zeile gesetzt. 
	\item {\bf Tabulator} \\
	Es wird immer wenn m\"oglich mit Tabulatoren einger\"uckt. Tabulatorbreite = 4 Leerzeichen. 
\end{itemize}