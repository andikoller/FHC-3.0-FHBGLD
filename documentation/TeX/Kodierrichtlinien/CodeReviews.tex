\chapter{Code Reviews}
\underline{{\bf 20. Mai 2011 - abschlusspruefung.class.php}}
\begin{itemize}
	\item Getter und Setter
		\begin{itemize}
			\item {\bf Wichtige} Attribute protected
			\item Realisierbar mit {\bf \_\_set} und {\bf \_\_get}
		\end{itemize}
\end{itemize}
\begin{verbatim}
public function __set($name, $value)
{
   $this->vars[$name] = $value;
}
\end{verbatim}
\begin{itemize}
	\item Klasse soll alles erledigen
		\begin{itemize}
		\item Konstruktor setzt \$new = true
		\item Funktion load() setzt \$new = false
		\item Save() unterscheidet true oder false (insert oder update)
		\end{itemize}
\end{itemize}
\begin{itemize}
	\item Allgemein
		\begin{itemize}
		\item Immer require\_once() verwenden
		\item Im Skriptkopf nur den Autor vermerken, der File wirklich bearbeitet hat
		\item \"Uber Klasse Kommentarblock \newline
		\end{itemize}
\end{itemize}
\underline{{\bf 10. Juni 2011}}
\begin{itemize}
	\item Allgemein
		\begin{itemize}
		\item Ordnerstruktur verbessern		
			\begin{itemize}
			\item local/cms Dateien in -> local/[sprache] verschieben
			\end{itemize}
		\item Phrasenmodul	
			\begin{itemize}
			\item Gro\ss - Kleinschreibung bei Name z.B.: ['global/fehlerBeiDerParameteruebergabe']
			\item Keine Sonderzeichen in Name
			\end{itemize}
		\item Variablen	
			\begin{itemize}
			\item Kein \_ vor private oder protected Variablen
			\item Docblock Kommentare bei Variablen nicht ntig, Kommentar mit {\bf Datentyp in Datenbank} reicht
			\end{itemize}
		\end{itemize}
\end{itemize}
\newpage

\underline{{\bf 07. Oktober 2011 - rdf.class.php}}
\begin{itemize}
	\item SOAP
		\begin{itemize}
		\item username / passwort optional in WSDL
		\item Objekt Array -> Complextype wenn mehere Parameter \"ubergeben werden 
		\end{itemize}
	\item XUL
		\begin{itemize}
		\item menuitemID = Element-Modul z.B.: menuitem-projekt
		\item Die Trennung erfolgt mit -, da beim markieren davor und dannach abgebrochen wird
		\end{itemize}
	\item RDF
		\begin{itemize}
		\item kein {\bf /alle} mehr -> sondern http://www.technikum-wien.at/[Datenbankname]
		\end{itemize}
	\item Adressoperator \&
	\item statische Variablen
	\item Configs s\"aubern und global.conf [include] f\"ur globale Konfiguration bereitstellen \newline
\end{itemize}

\underline{{\bf 11. November 2011 - bestellung.php - Projekt zu Bestellung zuordnen}}
\begin{itemize}
	\item Anf\"uhrungszeichen bei echo
		\begin{itemize}
		\item Es sollen die Einfachen Anf\"uhrungszeichen verwendet werden ['] und zur Stringverkettung der Beistrich [,]
			\begin{verbatim}
			echo 'Username: ',$user,' eingeloggt'; 
			\end{verbatim}
		\end{itemize}
	\item Einfache Apostroph ['] in String escapen -> htmlentities => eigene Funktion bauen
	\item Magic Quotes -> deprecated
	\item Session Authentifizierung im CIS / Kalenderschnittstellen
	\item Cross Side Scripting Schw\"achen ausarbeiten \newline
\end{itemize}

\underline{{\bf 16. Dezember 2011 - addslashes - String escapen}}
\begin{itemize}
	\item Einfache Anf\"uhrungszeichen bei echo und doppelte f\"ur Eigenschaften 
	\item Eine funktion db\_null\_value soll in die DB Klasse implementiert werden 
	\begin{itemize}
		\item Bei einem leeren String wird null geschrieben
		\item es soll ein Parameter \$quote \"ubergeben werden(defaultwert true) -> wenn false wird nicht gequotet (f\"ur Integer) 
	\end{itemize}
	\item Daten die aus DB kommen, sollen mit escaped werden(doublequotes ["] werden escaped) -> Quellcode soll nicht ver\"anderbar werden - Hierf�r steht in der Basisklasse die funktion convert\_html\_chars zur Verf�gung
	\item ! Auch quoten bei Integerzahlen in DB query? !!\newline
\end{itemize}

\underline{{\bf 10. Februar 2012 Escapen von DB Parametern - functions.inc.php ausmisten }}
\begin{itemize}
	\item Zum Escapen von DB Parametern steht nun eine neue Funktion zur Verf�gung: \$db->db\_add\_param(\$var, \$type, \$nullable) 
	\item Boolean Parameter ueber DB-Klasse parsen, damit keine Probleme mit anderen Datenbanken auftreten. (zB t/f bei Postgres 0/1 bei MySQL)
	\begin{itemize}
		\item Boolean die aus der Datenbank gelesen werden, sollen durch die funktion \$db->db\_parse\_bool(\$var) geparst werden
		\item Boolean die in die Datenbank geschrieben werden, sollen die Funktion \$db->db\_add\_param(\$var, \$type, \$nullable) verwenden
	\end{itemize}
	\item functions.inc.php s\"aubern
	\begin{itemize}
		\item Datumsfunktionen in datum.class.php auslagern
		\item eventuelle Auslagerung der Authentifizierungsfunktionen in eine eigene Klasse\newline
	\end{itemize}
\end{itemize}
\underline{{\bf 09. August 2012 Strichpunkt bei SQL Statements }}
\begin{itemize}
		\item SQL Statements im Code sind mit Strichpunkt am Schluss abzuschlie\ss{}en
\end{itemize}

\underline{{\bf 08. November 2012 - XSS L\"ucken und WYSIWYG}}
\begin{itemize}
	\item XSS-L\"ucke im CIS wo es m\"oglich ist fremde Webseiten in den main Frame einzuschleusen
		\begin{itemize}
		\item L\"osung: \"Uberpr\"ufung ob APP\_ROOT in URL vorkommt - wenn nicht dann Seite nicht laden und Email an uns senden
		\end{itemize}
	\item In den LV Infos k\"onnen Tags wie <ul><li> eingegeben werden -> soll \"uberarbeitet werden
	\item Anstatt dass jedes Jahr das gesamte CIS ins Archiv kopiert wird, sollen nur mehr die LV Infos weggespeichert werden
	\item Im WYSWYG Editor können Tags eingegeben werden. Auch <script> Tags
		\begin{itemize}
		\item SaveHTML dar\"uberlaufen lassen
		\item vor Speichern in DB durch Methode Tags rausfiltern
		\end{itemize}
	\item EMail Signaturen sollen in Zukunft als Phrase ausgelagert werden -> Auch für Community leichter zum Editieren \newline
	\item N\"achstes Codereview: Basis Klasse und Stundenplanklasse (Zusammenhang zu anderen Klassen)
\end{itemize}
