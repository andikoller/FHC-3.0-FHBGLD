\chapter{Statistik}
\label{Statistik}
Die vorhandenen Statistiken werden im Vilesci unter dem Men�punkt Auswertung
angezeigt.\\
Hier werden die Statistiken und zugeh�rigen Beschreibungen angezeigt.\\
\section{Statistiken hinzuf�gen}
Unter dem Punkt Stammdaten->Statistik k�nnen neue Statistiken hinzugef�gt
werden.\\
Statistiken k�nnen entweder auf eine URL zeigen (zB PHP Script) oder direkt als
SQL Befehl definiert werden.\\ In beiden F�llen k�nnen Variablen verwendet
werden:\\
''SELECT * FROM public.tbl\_person WHERE nachname like '\$filter' ''\\
oder\\
''../../content/statistik/bewerberstatistik.php?stsem=\$Studiensemester''\\
\\
Der Pfad bei URLs ist immer ausgehend vom Verzeichnis /vilesci/statistik/\\
\\
Beim Anzeigen der Statistik werden in diesem Fall Eingabefelder f�r die
entsprechenden Variablen angezeigt.\\
\\
Die Beschreibung von Statistiken werden �ber das CMS System angelegt. Diese
k�nnen �ber das Feld ContentID mit der Statistik verkn�pft werden.\\
\\
Soll eine Statistik nur einer ausgew�hlten Benutzergruppe zur Verf�gung stehen,
kann eine Statistik mit einer Berechtigung verkn�pft werden.
Wenn diese eintgetragen ist, ist die Statistik nur f�r Personen sichtbar, welche
diese Berechtigung auch besitzen.