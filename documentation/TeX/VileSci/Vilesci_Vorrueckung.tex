\chapter{Vorr�ckung}
\label{Vorr�ckung}
\section{Lehreinheiten}
Um die Lehreinheiten nicht jedes Semester neu Anlegen zu m�ssen, k�nnen diese
aus dem Vorjahr �bernommen werden.\\
Dies geschieht im Vilesci unter dem Men�punkt
Wartung->Vorr�ckung->Lehreinheiten.
Nach Auswahl der Daten, werden die Lehreinheiten, Gruppenzuordnungen und
Lektorenzuordnungen in das entsprechende Studiensemester kopiert. Es m�ssen dann
nur noch etwaige Anpassungen f�r das jeweilige Studiensemester durchgef�hrt
werden.\\
Die Vorr�ckung der Lehreinheiten erfolgt in der Regel nach der BIS-Meldung.
\section{Studierende}
Wenn ein Semester abgeschlossen ist, m�ssen die Studierenden in das n�chste
Semester vorger�ckt werden. Dies erfolgt �ber dem Men�punkt
Wartung->Vorr�ckung->Studenten.\\
Der obere Teil des Formulars dient nur zur Steuerung der Anzeige. Bei Auswahl
der Eintr�ge wird automatisch die angezeigte Liste aktualisiert.
Im unteren Teil des Formulars wird festgelegt welches Ausbildungssemester wohin
vorger�ckt wird. Die Auswahl des Ausbildungssemesters bleibt in der Regel auf
''alle'' gesetzt. Durch einen klick auf Vorr�cken, werden die Studierenden in
die entsprechenden Semester verschoben.\\
\\
Durch die Vorr�ckung wird, bei den in der Liste angezeigten Personen, ein neuer
Studentenstatus f�r das n�chste Semester hinzugef�gt. Dabei wird das aktuelle
Semester ermittelt und um 1 erh�ht. Sollte das neue Semester gr��er als
maxsemester des Studiengangs sein, wird das Semester auf maxsemester gesetzt.
Es werden nur aktive Personen vorger�ckt\\
\\
Voraussetzungen f�r die Vorr�ckung:\\
- Abbrecher m�ssen eingetragen und deaktiviert sein\\
- Absolventen m�ssen eingetragen und deaktiviert sein\\
- Diplomanden m�ssen als solche eintgetragen sein\\
- Incoming die nicht mehr vor Ort sind m�ssen deaktiviert sein\\
\\
Die Vorr�ckung der Studierenden erfolgt in der Regel nach Abschluss des
Semesters.\\
\\

\info{\textbf{Hinweis} Nach der Vorr�ckung befinden sich die Studierenden
weiterhin im alten Semester. Diese werden erst dann in das n�chste Semester veschoben, wenn der Beginn des
kommenden Studiensemesters n�her ist, als das Ende des vorigen.
}\\