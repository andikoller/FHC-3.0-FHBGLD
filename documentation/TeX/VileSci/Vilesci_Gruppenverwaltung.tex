\chapter{Gruppen}
\label{Gruppen}
\section{Grundlagen}
Im Vilesci k�nnen Gruppen auf 2 verschiedene Wege angelegt werden. Unter
Lehre->Gruppenverwaltung und unter Personen->Gruppen.\\
\\
Gruppen f�r den normalen Studiengebrauch legen wir �ber den Men�punkt Lehre
Gruppenverwaltung an. Hier k�nnen Lehrverbands und Spezialgruppen gleicherma�en
angelegt werden.\\
\\
Unter Personen->Gruppen k�nnen nur Spezialgruppen gewartet werden. Zus�tzlich
k�nnen hier Personen zu den Gruppen hinzugef�gt und entfernt werden.
\section{Mailverteiler}
Gruppen bei denen das Attribut ''Mailverteiler' gesetzt ist, sind automatisch
als Mailverteiler verf�gbar.\\
Wenn das Attribut ''generiert'' gesetzt ist, k�nnen die Teilnehmer dieser
Gruppe nicht h�ndisch gewartet werden. Die Teilnehmer in dieser Gruppe werden
Aufgrund von Funktionen oder anderen Attributen automatisch generiert.\\ 
\section{Zugangskontrolle}
Gruppen bei denen das Attribut ''ContentVisible'' gesetzt ist, k�nnen im CMS zur
Zutrittskontrolle verwendet werden. n�here Infos dazu finden Sie im CMS-Handbuch
