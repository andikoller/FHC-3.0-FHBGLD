\documentclass{scrbook}

\usepackage{ifpdf} 
\newif\ifpdf
\ifx\pdfoutput\undefined
	\pdffalse              	%normales LaTeX wird ausgefuehrt
\else
	\pdfoutput=1           
	\pdftrue               	%pdfLaTeX wird ausgefuehrt
\fi


\ifpdf
	%\usepackage{ae}        % Benutzen Sie nur
	%\usepackage{zefonts}  	% eines dieser Pakete
\else
	%%Normales LaTeX - keine speziellen Fontpackages notwendig
\fi

\ifpdf %%Einbindung von Grafiken mittels \includegraphics{datei}
	\usepackage[pdftex]{graphicx} %%Grafiken in pdfLaTeX
\else
	\usepackage[dvips]{graphicx} %%Grafiken und normales LaTeX
\fi


\ifpdf
	\pdfinfo
	{
    /Author (Andreas �sterreicher)                                
    /Title (Vilesci)     
    /Subject (Vilesci Handbuch)                                    
    /Keywords (Vilesci FH-Complete FH Technikum-Wien)
	}
\else			
\fi

\usepackage[pdftex,colorlinks=true,urlcolor=blue,linkcolor=blue]{hyperref}
\usepackage[ngerman]{babel} %Deutsche Worttrennung		
\usepackage[T1]{fontenc} % T1-Kodierung damit Umlaute richtig dargestellt werden
\usepackage[latin9]{inputenc} % latin9 Encoding
\usepackage{makeidx} % Stichwortverzeichnis erstellen aus \index{} Eintr�gen
\usepackage{float}
\usepackage[small,bf]{caption}
\usepackage{fancyhdr} % Paket zur Manipulation von Kopf und Fusszeile
\usepackage{import} % fuer subimport
\addtokomafont{chapter}{\color[rgb]{0.0,0.376,0.584}}
\addtokomafont{section}{\color[rgb]{0.0,0.376,0.584}}
\addtokomafont{subsection}{\color[rgb]{0.0,0.376,0.584}}

\renewcommand{\rmdefault}{phv} % Arial
\renewcommand{\sfdefault}{phv} % Arial

\makeindex

\graphicspath{{../../images/}}

\setlength{\tolerance}{2000} % Steuerung von Zeilenumbruch
\setlength{\parindent}{0pt} % Einr�cktiefe f�r Abs�tze
\setlength{\parskip}{1ex plus 0.5ex minus 0.2ex}
\addtolength{\textheight}{2cm}
\addtolength{\headheight}{2pt}
\setlength{\captionmargin}{20pt}
\floatstyle{plain}
\floatname{example}{Example}

\newfloat{example}{hbtp}{loe}[chapter]
\floatplacement{figure}{hbtp}
\floatplacement{table}{htbp}

\newcommand{\dollar}{\char36}

% Hinweisbox mit Rufzeichen
\newenvironment{info}[1]{
    \hspace{-10mm}
    \fbox{
        \begin{minipage}{1cm}
        \includegraphics[width=1cm]{icon_info}
        \end{minipage}
        \begin{minipage}{14.5cm}
        #1
        \end{minipage}
    }
}

% Hinweisbox mit Warndreieck
\newenvironment{achtung}[1]{
    \hspace{-10mm}
    \fbox{
        \begin{minipage}{1cm}
        \includegraphics[width=1cm]{icon_achtung}
        \end{minipage}
        \begin{minipage}{14.5cm}
        #1
        \end{minipage}
    }
}

% Hinweisbox mit Stoptafel
\newenvironment{halt}[1]{
    \hspace{-10mm}
    \fbox{
        \begin{minipage}{1cm}
        \includegraphics[width=1cm]{icon_halt}
        \end{minipage}
        \begin{minipage}{14.5cm}
        #1
        \end{minipage}
    }
}

% Hinweisbox mit Gl�hbirne
\newenvironment{idee}[1]{
    \hspace{-10mm}
    \fbox{
        \begin{minipage}{1cm}
        \includegraphics[width=1cm]{icon_idee}
        \end{minipage}
        \begin{minipage}{14.5cm}
        #1
        \end{minipage}
    }
}


\setlength{\unitlength}{1mm}

% Marker f�r Bilder
\newenvironment{markier}[5]{
    
    \thicklines \put(#2,#3){\vector(#4,#5){5}} \thinlines
    \put(#2,#3){\circle*{5}}
    \put(#2,#3){\textcolor{black}{\circle{5}}\makebox(-10,0){\textcolor{white}{#1}}}
}

\begin{document}

\ifpdf
	\DeclareGraphicsExtensions{.pdf,.jpg,.png}
\else
	\DeclareGraphicsExtensions{.eps}
\fi

\pagestyle{fancyplain}
% Titelseite einbinden
%
% Titelseite, Abstrakt, Danksagung und Inhaltsverzeichnis
%
%% eigene Titelseitengestaltung %%%%%%%%%%%%%%%%%%%%%%%%%%%%%%%%%%%%%%%    

\begin{titlepage}
\begin{center}
\vspace*{40mm} \huge Vilesci-Handbuch\\
\vspace*{10mm}
\large \textsc{FH-Complete Vilesci Handbuch}

\vfill \includegraphics[width=130mm]{vilesci}
	
\vfill \textsc{FH Technikum Wien}\\

Wien, \today
\end{center}
\end{titlepage}

\tableofcontents			% Inhaltsverzeichnis
\frontmatter				% Vorspann
\chapter{Einleitung}
\mainmatter					% Hauptteil

%% Kapitel Anfang %%%%%%%%%%%%%%%%%%%%%%%%%%%%%%%%%%%%%%%%%%%%%%%%%
\chapter{Statistik}
\label{Statistik}
Die vorhandenen Statistiken werden im Vilesci unter dem Men�punkt Auswertung
angezeigt.\\
Hier werden die Statistiken und zugeh�rigen Beschreibungen angezeigt.\\
\section{Statistiken hinzuf�gen}
Unter dem Punkt Stammdaten->Statistik k�nnen neue Statistiken hinzugef�gt
werden.\\
Statistiken k�nnen entweder auf eine URL zeigen (zB PHP Script) oder direkt als
SQL Befehl definiert werden.\\ In beiden F�llen k�nnen Variablen verwendet
werden:\\
''SELECT * FROM public.tbl\_person WHERE nachname like '\$filter' ''\\
oder\\
''../../content/statistik/bewerberstatistik.php?stsem=\$Studiensemester''\\
\\
Der Pfad bei URLs ist immer ausgehend vom Verzeichnis /vilesci/statistik/\\
\\
Beim Anzeigen der Statistik werden in diesem Fall Eingabefelder f�r die
entsprechenden Variablen angezeigt.\\
\\
Die Beschreibung von Statistiken werden �ber das CMS System angelegt. Diese
k�nnen �ber das Feld ContentID mit der Statistik verkn�pft werden.\\
\\
Soll eine Statistik nur einer ausgew�hlten Benutzergruppe zur Verf�gung stehen,
kann eine Statistik mit einer Berechtigung verkn�pft werden.
Wenn diese eintgetragen ist, ist die Statistik nur f�r Personen sichtbar, welche
diese Berechtigung auch besitzen.
\chapter{Wiederholer}
\label{Wiederholer}
Studierende k�nnen bei negativem Abschluss eines Semesters dieses wiederholen.
Dazu wird der Studierende normale vorger�ckt. Nach der Vorr�ckung, muss der
Status im Prestudent-Karteireiter im FAS korrigiert werden, und der Studierende
ins entsprechende Semester zur�ckverschoben werden.\\
\\
Ein Spezialfall ergibt sich, wenn der Studierende bereits als Abbrecher
BIS-gemeldet wurde, und sich erst danach dazu entschlie�t, das Semester zu
wiederholen. In diesem Fall, darf KEIN neuer Status zum alten
Studierendeneintrag hinzugef�gt werden. Die Person muss neu inskribiert werden.
Diese Art von Wiederholern kann im Viesci unter dem Men�punkt Personen->Stg
Wiederholer angelegt werden. Hier wird nach dem Namen der Person gesucht und
danach das Ausbildungssemester sowie das Studiensemester ausgew�hlt in dem er
wieder einsteigen m�chte. Er wird dann automatisch im FAS als Interessent f�r das Semester
eingetragen. 

\chapter{Doppelte Personeneintr�ge zusammenlegen}
\label{Doppelte Personeneintr�ge zusammenlegen}
Wenn eine Person in mehreren Studieng�ngen studiert, wird nur ein
Personeneintrag angelegt, jedoch mehrere Studententeintr�ge.
Sollte dies beim Anlegen von Personen �bersehen werden, k�nnen die
Personendatens�tze im nachhinein zusammengelegt werden.\\
\\
Dies geschieht im Vilesci unter dem Men�punkt Personen->Zusammenlegen.
Es kann nach der Person gesucht werden. Danach werden 2 gleiche Listen
angezeigt. In der linken Liste wird die Person markiert, welche gel�scht werden
soll, rechts jene die bestehen bleibt. Durch einen klick auf den Pfeil in der
Mitte werden die Personendatens�tze zusammengelegt.\\
\\
Bei diesem Vorgang, wird einer der Personendatens�tze gel�scht. Die Adressen,
Kontakte, Betriebsmittel, etc werden mit dem anderen Personendatensatz
verkn�pft. Danach sollten im FAS die eventuell doppelt vorhandenen Kontaktdaten
und Adressen gel�scht werden.\\
\\
\\
\achtung{\textbf{Achtung} Personen die einmal zusammengelegt wurden, k�nnen
nur schwer wieder getrennt werden. Kontrollieren sie ihre Auswahl vor dem
zusammenlegen noch einmal!}\\


\chapter{Vorr�ckung}
\label{Vorr�ckung}
\section{Lehreinheiten}
Um die Lehreinheiten nicht jedes Semester neu Anlegen zu m�ssen, k�nnen diese
aus dem Vorjahr �bernommen werden.\\
Dies geschieht im Vilesci unter dem Men�punkt
Wartung->Vorr�ckung->Lehreinheiten.
Nach Auswahl der Daten, werden die Lehreinheiten, Gruppenzuordnungen und
Lektorenzuordnungen in das entsprechende Studiensemester kopiert. Es m�ssen dann
nur noch etwaige Anpassungen f�r das jeweilige Studiensemester durchgef�hrt
werden.\\
Die Vorr�ckung der Lehreinheiten erfolgt in der Regel nach der BIS-Meldung.
\section{Studierende}
Wenn ein Semester abgeschlossen ist, m�ssen die Studierenden in das n�chste
Semester vorger�ckt werden. Dies erfolgt �ber dem Men�punkt
Wartung->Vorr�ckung->Studenten.\\
Der obere Teil des Formulars dient nur zur Steuerung der Anzeige. Bei Auswahl
der Eintr�ge wird automatisch die angezeigte Liste aktualisiert.
Im unteren Teil des Formulars wird festgelegt welches Ausbildungssemester wohin
vorger�ckt wird. Die Auswahl des Ausbildungssemesters bleibt in der Regel auf
''alle'' gesetzt. Durch einen klick auf Vorr�cken, werden die Studierenden in
die entsprechenden Semester verschoben.\\
\\
Durch die Vorr�ckung wird, bei den in der Liste angezeigten Personen, ein neuer
Studentenstatus f�r das n�chste Semester hinzugef�gt. Dabei wird das aktuelle
Semester ermittelt und um 1 erh�ht. Sollte das neue Semester gr��er als
maxsemester des Studiengangs sein, wird das Semester auf maxsemester gesetzt.
Es werden nur aktive Personen vorger�ckt\\
\\
Voraussetzungen f�r die Vorr�ckung:\\
- Abbrecher m�ssen eingetragen und deaktiviert sein\\
- Absolventen m�ssen eingetragen und deaktiviert sein\\
- Diplomanden m�ssen als solche eintgetragen sein\\
- Incoming die nicht mehr vor Ort sind m�ssen deaktiviert sein\\
\\
Die Vorr�ckung der Studierenden erfolgt in der Regel nach Abschluss des
Semesters.\\
\\

\info{\textbf{Hinweis} Nach der Vorr�ckung befinden sich die Studierenden
weiterhin im alten Semester. Diese werden erst dann in das n�chste Semester veschoben, wenn der Beginn des
kommenden Studiensemesters n�her ist, als das Ende des vorigen.
}\\
\chapter{Gruppen}
\label{Gruppen}
\section{Grundlagen}
Im Vilesci k�nnen Gruppen auf 2 verschiedene Wege angelegt werden. Unter
Lehre->Gruppenverwaltung und unter Personen->Gruppen.\\
\\
Gruppen f�r den normalen Studiengebrauch legen wir �ber den Men�punkt Lehre
Gruppenverwaltung an. Hier k�nnen Lehrverbands und Spezialgruppen gleicherma�en
angelegt werden.\\
\\
Unter Personen->Gruppen k�nnen nur Spezialgruppen gewartet werden. Zus�tzlich
k�nnen hier Personen zu den Gruppen hinzugef�gt und entfernt werden.
\section{Mailverteiler}
Gruppen bei denen das Attribut ''Mailverteiler' gesetzt ist, sind automatisch
als Mailverteiler verf�gbar.\\
Wenn das Attribut ''generiert'' gesetzt ist, k�nnen die Teilnehmer dieser
Gruppe nicht h�ndisch gewartet werden. Die Teilnehmer in dieser Gruppe werden
Aufgrund von Funktionen oder anderen Attributen automatisch generiert.\\ 
\section{Zugangskontrolle}
Gruppen bei denen das Attribut ''ContentVisible'' gesetzt ist, k�nnen im CMS zur
Zutrittskontrolle verwendet werden. n�here Infos dazu finden Sie im CMS-Handbuch

\subimport{../FAS/}{FASo_Lehrveranstaltung.tex}
\section{Erweitert}
\label{Lehrveranstaltung - Erweitert}
Wenn sich die Attribute einer Lehrveranstaltung Aufgrund einer
Curriculums�nderung �ndern, ist eine neue Lehrveranstaltung anzulegen.\\
Insbesondere bei �nderung der ECTS-Punkte oder Bezeichnung der Lehrveranstaltung
ist darauf zu achten, da sich eine �nderung auf alle bisherigen Zeugnisse
auswirkt.
\subsection{Berechnung der Semesterstunden}
Bei der Lehrveranstaltung sind immer die Semesterstunden der LV einzutragen.
Dies sollte nicht mit den SWS verwechselt werden welche am Zeugnis aufscheinen.
Die SWS berechnen sich automatisch aus den Semesterstunden durch die Anzahl der
Wochen des Studiengangs. Ist bei einem Studiengang kein Wert f�r die Wochen
angegeben, wird automatisch 15 als Standardwert genommen.\\ Die
Unterrichtswochen k�nnen pro Semester unterschiedlich sein. (Derzeit keine GUI
zur Wartung vorhanden).

\chapter{Ampel}
\label{Ampel}
Das Ampelsystem ist eine Art Erinnerungssystem f�r Mitarbeiter und Studierende.
Ampeln werden im CIS angezeigt und m�ssen dort best�tigt werden.
Fallbeispiele f�r Ampeln:\\
- Erinnerung zur Eintragung der Zeitw�nsche f�r Lektoren\\
- Best�tigung von Verordnungen (Brandschutzordnung etc)\\
\\
\section{Ampel hinzuf�gen}
Unter dem Punkt Stammdaten->Ampel k�nnen neue Ampeln erstellt
werden.\\
\\
Die folgenden Felder m�ssen bef�llt werden:\\
\begin{itemize}
	\item kurzbz - Kurze eindeutige Bezeichnung der Ampel
	\item benutzer select - SQL Select. Dieser liefert die UID aller Personen die die  Ampel betrifft
	\item beschreibung - Beschreibungstext
	\item deadline - Datum der Deadline
	\item vorlaufzeit - Anzahl der Tage vor der Deadline bei der die Ampel auf Gelb schaltet
	\item verfallszeit - Anzahl der Tage nach der Deadline nach der die Ampel verschwindet
\end{itemize}
\chapter{Infoscreen}
\label{Infoscreen}
Unter dem Punkt Stammdaten->Infoscreens k�nnen die Infoscreens verwaltet werden.\\
Infoscreens sind Bildschirme mit allgemeinen Informationen.\\
\\
\\
Die Infoscreens k�nnen unterschiedlichen Content anzeigen. Dazu muss der Infoscreen in der Verwaltungsseite angelegt werden. �ber die IP-Adresse wird der Infoscreen identifiziert.\\
\\
Sobald ein Infoscreen angelegt ist, kann diesem Content zugeordnet werden.\\
Folgende Felder k�nnen bef�llt werden:
\begin{itemize}
	\item InfoscreenID - ID des Infoscreens der den Content anzeigen soll. Wenn dieses Feld leer ist, dann wird die Seite auf allen Infoscreens angezeigt
	\item ContentID - ID des Contents aus dem CMS
	\item GueltigVon/GueltigBis - Datumsbereich innerhalb dessen der Content am Infoscreen angezeigt werden soll. Wenn dieses Feld leer ist, wird die Seite immer ber�cksichtigt.
	\item Refreshzeit - Legt fest, wie viele Sekunden diese Seite sichtbar ist
\end{itemize}
Die Seiten werden der Reihenfolge nach abwechselnd angezeigt.\\
Zus�tzlich zu den angegebenen Seiten werden die News angezeigt. Diese m�ssen nicht extra eingetragen werden.\\
\\
\achtung{\textbf{Achtung} Damit die Infoscreens korrekt funktionieren, muss der Browser des Infoscreens Cookies erlauben.}\\




%% Kapitel Ende   %%%%%%%%%%%%%%%%%%%%%%%%%%%%%%%%%%%%%%%%%%%%%%%%%
\appendix					% Beginn des Anhangs
\chapter{Schluss}
\listoftables				% Tabellenverzeichnis
\listoffigures				% Abbildungsverzeichnis
\end{document}