\chapter{Infoscreen}
\label{Infoscreen}
Unter dem Punkt Stammdaten->Infoscreens k�nnen die Infoscreens verwaltet werden.\\
Infoscreens sind Bildschirme mit allgemeinen Informationen.\\
\\
\\
Die Infoscreens k�nnen unterschiedlichen Content anzeigen. Dazu muss der Infoscreen in der Verwaltungsseite angelegt werden. �ber die IP-Adresse wird der Infoscreen identifiziert.\\
\\
Sobald ein Infoscreen angelegt ist, kann diesem Content zugeordnet werden.\\
Folgende Felder k�nnen bef�llt werden:
\begin{itemize}
	\item InfoscreenID - ID des Infoscreens der den Content anzeigen soll. Wenn dieses Feld leer ist, dann wird die Seite auf allen Infoscreens angezeigt
	\item ContentID - ID des Contents aus dem CMS
	\item GueltigVon/GueltigBis - Datumsbereich innerhalb dessen der Content am Infoscreen angezeigt werden soll. Wenn dieses Feld leer ist, wird die Seite immer ber�cksichtigt.
	\item Refreshzeit - Legt fest, wie viele Sekunden diese Seite sichtbar ist
\end{itemize}
Die Seiten werden der Reihenfolge nach abwechselnd angezeigt.\\
Zus�tzlich zu den angegebenen Seiten werden die News angezeigt. Diese m�ssen nicht extra eingetragen werden.\\
\\
\achtung{\textbf{Achtung} Damit die Infoscreens korrekt funktionieren, muss der Browser des Infoscreens Cookies erlauben.}\\

