\chapter{Vorr�ckung}
\label{vorrueckung}
Die Vorr�ckung ist die Vorbereitung auf das n�chste Studiensemester und wird nach Auftrag vom Studiengang vom Administrator durchgef�hrt. 
\section{Lehreinheiten}
Um die Planung f�r das neue Semester zu vereinfachen, k�nnen die Lehreinheiten des Vorjahres vorger�ckt werden. Dies ist speziell dann sinnvoll, wenn sich gegen�ber dem Vorjahr keine oder nur wenige �nderungen ergeben.
\section{Studenten}
Unter der Vorr�ckung der Studenten wird die Eintragung eines Status und der Lehrverbandgruppenzuteilung f�r das n�chste Studiensemester verstanden. Vorger�ckt werden nur Personen, die als \underline{aktiv} gekenntzeichnet sind.\\ 
\\
Die Vorr�ckung muss f�r jeden Studiengang jedes Semester durchgef�hrt werden, damit die Studierenden ins n�chste Semester aufsteigen.\\
Vor der durchf�hrung der Vorr�ckung sollten folgende Aktionen durchgef�hrt werden:\\
- Deaktivierung von Incoming die nicht mehr im Haus sind\\
- Status Absolvent setzen und deaktivierung von fertigen Studierenden \\
\\
Wiederholer eines Semesters werden normal vorger�ckt und nach der Vorr�ckung in das entsprechende Semester zur�ckgeschoben.\\
\\
\underline{Beispiel}:\\
Maria Musterhaft ist im Wintersemester 2006 Studentin im 1.Semester und der Gruppe 1A1 zugeteilt. Bei der Studentenvorr�ckung wird nun ein neuer Status Student im 2. Semester f�r das Sommersemester 2007 eingetragen. Weiters erfolgt eine Zuteilung der Studentin zu der Gruppe 2A1.\\
Die Semestereintragungen der Stati k�nnen nur im Bereich von 1 bis zur maximalen Semesteranzahl des Studiengangs sein (z.B. f�r Bachelorstudieng�nge sind das idR. 6 Semester). Die Semestereintragungen bei den Lehrverb�nden k�nnen auch 0 und Zahlen gr��er als die Semesterzahl sein, diese werden aber bei der Vorr�ckung dann nicht ver�ndert (z.B. wenn Unterbrecher in die Gruppe 0B verschoben wurden, befinden sich diese auch im n�chsten Semester dort).\\