\chapter{Aufgaben}
\section{Anlegen von Interessenten und Incoming}
\idee{\textbf{Hinweis} Ein Student kann kein zweites Mal in einem Studiengang angelegt werden!}\\

\begin{enumerate}
	\item Karteikarte \textit{Verband} im Listenfeld 1 ausw�hlen.
	\item In der Karteikarte \textit{Student} des Listenfeld 2 die Taste \includegraphics{icon_neu.png} anklicken.
	\item Es �ffnet sich die Eingabemaske.
	\item Eingabemaske bef�llten:
	\begin{itemize}
		\item Mit * markierte Felder m�ssen bef�llt bzw. ausgew�hlt werden.
		\item Handelt es sich hier um einen Incoming und keinen Interessenten, mu� die Checkbox \textit{Incoming} angekreuzt werden! \textbf{Achtung}, beim Anlegen des Incoming werden auch dessen Personenkennzeichen und UID erzeugt. Eine nachtr�gliche �nderung mit einigem Aufwand verbunden. 
	\end{itemize}
	\item Anklicken der Taste \textit{Vorschlag laden}:
	Wurden Vorname, Geburtsdatum oder beides eingegeben, werden die eingegebenen Daten in der Datenbank gesucht und, so es Treffer gibt, die Ergebnisse (Nachname, Vorname, Geburtsdatum, SVNR, Geschlecht, Adresse) rechts auf der Anzeige gezeigt. Durch Anklicken des entsprechenden Radiobuttons kann ein Ergebnis ausgew�hlt werden oder eine neue Person angelegt werden. Wird ein Ergebnis ausgew�hlt, werden die Stammdaten dieses Personeneintrags verwendet und m�ssen nicht neu eingegeben werden.
	\item Durch Dr�cken der Taste \textit{Speichern} werden die Daten in die Datenbank �bertragen.
\end{enumerate}
\section{�ndern des Status}
\section{Inskription}
\section{Buchungen und Zahlungen}
\section{Noten}
\subsection{Noteneingabe}
\subsection{Notenimport}
\subsection{Notentool}
\section{Lehrauftragsvergabe}
\begin{enumerate}
	\item Anlegen einer Lehreiheit
	\item Zuordnen der Lektoren:
	\item Zuordnen der Gruppen:
\end{enumerate}
\subsection{Wahlf�cher}
Es wird f�r alle Wahlf�cher je eine zugeh�rige Spezialgruppe angelegt, in die dann die teilnehmenden Studenten hineingezogen werden. Studenten anderer Studieng�nge k�nnen oft nicht direkt zugeordnet werden, da die Zugriffsrechte dies nicht erlauben. Studenten anderer Studieng�nge m�ssen mit der Suchfunktion im Listenfeld 2 gefunden werden und dann von dort in die entsprechende Spezialgruppe in Listenfeld 1 gezogen werden.

\achtung{\textbf{Wichtig} Die studiengangsfremden Studenten m�ssen in Spezialgruppen und auf keinem Fall in Lehrverbandsgruppen (z.B. 1A1, 5B3, etc.) gezogen werden, da es sonst zu Problemen im Herkunftsstudiengang kommt!}\\

\subsection{Incoming}
Incoming-Studenten werden in ihrem Studiengang im Semester 0I angelegt und dort einzeln in Untergruppen (0I1, 0I2,...) aufgeteilt. Die Untergruppe eines Incoming-Studenten wird nun einer Lehreinheit der vom Studenten ausw�hlten Lehrveranstaltung zugeordnet.